\documentclass[10pt]{article}
\usepackage{hyperref}

\title{Plan for Automatic Program Analysis project 1}
\author{Eduard van der Bent \and Frank Dedden \and Wilco Kusee \and Falco Peijnenburg}


% Moet nogal wat tekst bijkomen denk ik, maar wat precies?

\begin{document}
\maketitle

\section*{Description}
We will be analysing a Lua program for dead code, i.e. code that will never be executed. We will not cover the complete language of Lua, as this is far too much work, however we will at least be supporting the following language features:
\begin{itemize}
	\item Conditionals
	\item Loops
	\item Functions
	\item Tables
	% Nog verder uitbreiden?
\end{itemize}
Which we can extend with the following:
\begin{itemize}
	\item Higher-order functions
	\item Unknown functions
\end{itemize}
To achieve this, we will first parse a piece of Lua source code. Both the parsing and analysing will be done in Haskell and the uuagc Attribute Grammar system. The parser has already been written by Falco Peijnenburg and can be used as is. We will extend upon this to analyse the input code. The parser can be found here:

\url{https://github.com/FPtje/GLuaParser}

\end{document}
